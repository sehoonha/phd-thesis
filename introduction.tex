%1234567890123456789012345678901234567890123456789012345678901234567890123456789
\chapter{Introduction}
% Our problem (=dynamic motor skills) is important
Performing highly dynamic motions with agility and grace has been one of the
greatest challenges for athletes, game characters, and robots.
The philosophy of Parkour, ``an art of overcoming obstacles as swiftly and
efficiently as possible using only your body, '' shows the importance of
dynamic motor skills in two aspects: dynamic motions can be both artistic 
methods of self-expression and efficient mechanisms of transportation. 
In fact, it has been a great milestone in the various research area
to develop physics-based controllers that can execute natural 
and agile motions.
For instance, dynamic motion controllers are developed 
for game characters to generate live interactive behaviors or optimal 
motions that minimize energy consumption.
In robotics, dynamic motor skills allow robots to overcome obstacles and
to move in uneven terrains, which can be often located in 
the disaster places.
To maximize capabilities of virtual characters and robots for 
further challenging motor tasks, it is inevitable to develop 
physic-based dynamic motion controllers.

% I will solve virtual and real both, to get the benefits
Virtual characters and real robots are two main subjects of motor control
problems in computer graphics and robotics.
Although they have different applications and limitations,
it is also true that control problems of both subjects have common
properties, such as non-linearity of the objective function, 
under-actuated characters, high-dimensional control parameters, 
and discontinuity due to the contacts.
Therefore, principles and algorithms developed in one domain often can be
transferred to the other domain.
For instance, many optimization algorithms for finding the best control
parameters, such as PEGASUS and CMA-ES, have been successfully applied
to the virtual characters and robots.
Further, a virtual simulation of a robot is often used as a testbed for
developing hardware compatible controllers due to the expensive cost and
time-consuming trials.
In this dissertation, I will discuss control of both virtual and real humanoids
by demonstrating the different problem formulation and solutions, and further
proposing the optimization technique for hardware that exploits the experience
in the virtual simulation.

% The first difficulty: dynamic motion
However, developing effective and robust controllers for dynamic motor skills 
is a very difficult and time-consuming task, which requires a lot of manual
efforts and computational resources.
First, dynamic motor skills are 
characterized by abrupt accelerations and decelerations of momentum,
frequent changes of contacts, and explosive usage of torques near limitations.
Due to these attributes, controllers must be able to generate efficient
torque trajectory under constraints and handle a wide range of initial
conditions for robustness.
One great example of challenging motor skills is a falling motion
of humanoids, because it accompanies huge changes of vertical momentum within 
a very short time window.
In addition, it is a fundamental motor skill that protects the subject from 
severe injuries and connects the previous and next actions for fluent
transitions.
Therefore, the development of controllers for falling and landing motions
will make characters and robots to operate safely in the unexpected environment,
and its principles can be applied to the other highly dynamic motions
with huge momentum.

% The first difficulty: a design and our approach
However, developing effective and robust controllers is a very difficult and
time-consuming task, which requires a lot of manual efforts and computational
resources.
First, a design process for controllers require a lot of prior knowledge
including mechanisms, target poses, momenta, and contacts.
One promising direction to acquire prior knowledge is a paradigm of
``learning from demonstration''[], but it still requires expensive motion
captured reference motions and costly reconstruction processes [].
On the other hand, a learning framework that can directly train policies
without reference motion will be 

In this dissertation, direct training of a virtual character does not require 
reference motions but often needs optimizations with the manually
designed parametrization and a

% The second difficulty (optimization) and our approach
Another difficulty arises when optimizing the parameters for the controllers.
Usually, whole-body dynamic tasks typically have a cost function that is 
multimodal and very non-smooth due to the combined challenge of contact 
switches, dynamic balance, and high-dimensional space.
Further, most control parameters will lead to undesired behaviors such as
losing balance or unexpected contacts.
These difficulties often requires the most robust optimization algorithm, 
CMA-ES, which is designed for non-linear non-convex black-box
optimizations.
In addition, training policies for hardware adds another problem.
Although a simulator is a useful tool that provides fast evaluations of the
controllers, but controllers turned in simulation do not work on hardware due
to the differences in two systems, so called simulation bias.
One way to alleviate simulation bias is iteratively updating the simulation
model [,,], so called model-based policy search, which is an on-going topic in
robotics.


% Reformulate as the third difficulty (simulation bias)
In fact, a dynamic motion controller for humanoids has been
a popular topic in many research areas, including computer graphics,
robotics, and bio-mechanics.
In these fields, there are two types of subjects for physics-based controllers
: a characters in the simulation and a humanoid with read hardware.
Since the control problems of virtual and real humanoids have shared 
properties, such as non-linearity, high-dimension, and discontinuity, 
algorithms and principles developed in one domain can be transferred to 
the other domain.
However, different goals and assumptions often lead to
slightly different approaches.
The primary goal of the physics-based control in computer animation is to
generate natural and realistic human behaviors in the virtual world.
For this goal, a wide range of controllers have been developed for highly
dynamic stunts in the level of athletes, such as leaping, vaulting, and
flipping, under the assumption that a character has enough agility to execute
these motions.
On the other hand, a humanoid controller in the robotics must be compatible 
with the hardware, which generally has more strict joints, torques, and power
limitations.
Furthermore, the robustness of the controller becomes a more important issue
because a failed motion can potentially cause detrimental damage to the robot
and enormous cost to repair.
% The third difficulty: simulation bias and our approach
- Although virtual and real are similar, they are different
- Difference may arise in many different places, such as ...
- Model-based learning is our approach

% Identify three categories problems.
In this dissertation, following problems are identified for developing 
controllers for highly dynamic motor skills.
- A falling motion is selected, due to the change of momentum
- Further, learning framework is presented for efficient training
- Finally, 

\section{Falling Strategies for Humanoids}
% Introduction on the falling - Motivation, Description, Product, Goal
Highly dynamic motions often accompany the abrupt momentum changes, which can
cause large contact forces to characters.
Therefore, how to manage falls is a fundamental motor skill to reduce damages
to humanoids and achieve fluent transitions between motor skills.
I will present two falling strategies for falls of virtual and real humanoids,
which are inspired by falling of Traceurs (Parkour practitioners) and robots.
The effectiveness of the presented strategies will be validated in physics
simulation, and experimentally tested on a small-size humanoid.

\subsection{Falling and Landing Motion Control for Virtual Characters}
% Goal, Method (Strength), Verification + Image
In Chapter 3, I will show how to create an on-line controller for generating 
agile and natural falling motions of the virtual character that can land from 
various heights and velocities.
Inspired by falling skills of Parkour, I will formulate the falling scenario
with three phases, airborne, impact, and rolling based on the contact states 
of the character.
Two controllers are designed for airborne and rolling phases and a regression
analysis is conducted to find an optimal impact angle that can connect two
controllers.
Finally, I will demonstrate that the algorithm induces smaller joint stress,
which is still four times lower than a rag-doll motion at the worst cases.

\subsection{Multiple Contact Planning for Humanoids}
% Goal, Method (Strength), Verification + Image
Chapter 4 will provide an introduction on a multiple contact planning to
reduce the damage to the real humanoid, which provides a unified framework
for the existing falling techniques [,,].
Then, I will show how to efficiently optimize the multiple contact falling
strategy to the given initial state using a simplified model and dynamic 
programming.
Finally, various scenarios will be tested on simulated humanoids and the
actual hardware to show that our algorithm plans various falling strategies
with different contact sequences.

\section{Learning of Dynamic Controller for Characters}
% Introduction on the learning - Motivation, Description, Product, Goal
Teaching a physically simulated character a new motor skill entirely depends 
on the effort of the controller designer, from the design of the control 
architecture to the tweaking of low-level control parameters.
To simplify the learning process, we introduce an intuitive and 
interactive framework for developing dynamic controllers that is inspired by
how humans learn dynamic motor skills through progressive process of coaching
and practices.
Further, we propose two optimization techniques that can expedite the convergence and generalize the skill to the parametrized objectives.

\subsection{Iterative Design of Dynamic Controllers}
% Goal, Method (Strength), Verification + Image
In Chapter 5, I will describe a general framework to design dynamic
controllers using high-level, human-readable instructions.
In this framework, we introduce a “control rig” that can control the multiple
joints at the same time, which can be easily created by a set of high level
instructions.
The details of controllers and objectives formulation using control rigs and
instructions will be shown with examples.

\subsection{Optimization with Failure Learning}
% Goal, Method (Strength), Verification + Image
A controller with many user constraints is difficult to optimize due to the
relatively small feasible region.
In Chapter 6, I will describe a new optimization algorithm based on the
observation of human’s ability to learn from failure.
The proposed algorithm, CMA-C (Covariance Matrix Adaptation with
Classification) utilizes the failed simulation trials to approximate 
an infeasible region in the space of control rig parameters, resulting a
faster convergence than the standard CMA-ES.

\subsection{Optimization for Parametrized Motor Skills}
% Goal, Method (Strength), Verification + Image
In Chapter 7, I will show how to optimize parametrized motor skills that are
essential for autonomous robots operating in an unpredictable environment. 
By evolving a parametrized probability distribution, the algorithm reduces 
the number of samples required to optimize a parametrized skill for 
all the tasks in the range of interest. 

\section{Model-based Learning for virtual and real characters}
% Introduction on the bias

\section{Contributions}
The control and optimization methods discussed in this dissertation provide
several contributions to the computer animation community. 
These contributions are as follows:


\begin{itemize}
\item \textbf{A falling and landing strategy for various initial conditions}
  The algorithm presented in the dissertation allows the character to fall from 
  a wide range of heights and initial speeds, continuously roll on the ground, 
  and get back on its feet, without inducing large stress on joints at any
  moment.
\item \textbf{A multiple contact falling strategy for humanoids}
  We introduce a new planning algorithm to minimize the damage of humanoid 
  falls by utilizing multiple contact points. 
\item \textbf{An iterative learning framework for dynamic motor skills}
  Inspired by how humans learn dynamic motor skills through progressive process 
  of coaching and practices, we introduce an intuitive and interactive 
  framework for developing dynamic controllers. 
\item \textbf{An optimization technique that utilized failed samples}
  We introduce a novel efficient optimization algorithm, CMA-C, that shows 
  that faster convergence rate by approximating the infeasible region of a  
  particular type of failure with Supported Vector Machines.
\item \textbf{An optimization technique for parametrized objectives}
  We introduces an efficient evolutionary optimization algorithm for learning
  parametrized skills to achieve whole-body dynamic tasks. 
\end{itemize}

