%1234567890123456789012345678901234567890123456789012345678901234567890123456789

%% Demonstrating strength and agility has been a great \revised{milestone} for
%% virtual and real humanoids.
\revised{Demonstrating strength and agility on virtual and real humanoids has
  been an important goal in computer graphics and robotics.
  However, developing physics-based controllers for various agile motor skills
  requires a tremendous amount of prior knowledge and manual labor due to its
  complex mechanisms.}
%% For example, agile motions of virtual characters are popular content to 
%% captivate consumers of computer animations and games.
%% Also, real humanoids require enough agility to efficiently execute various
%% tasks which are difficult and dangerous to be done by human workers.
The focus of the dissertation is to develop a set of computational tools to
expedite the design process of physics-based controllers that can execute a
variety of agile motor skills on virtual and real humanoids.
Instead of directly deploying motions on real humanoids, this dissertation takes
an approach that develops appropriate theories and models in virtual simulation
and \revised{systematically} transfers the solutions to hardware systems.

The algorithms and frameworks in this dissertation span various topics from
specific physics-based controllers to general learning frameworks.
We first present an online algorithm for controlling falling and landing
motions of virtual characters.
The proposed algorithm is effective and efficient enough to generate falling
motions for a wide range of arbitrary initial conditions in real-time.
Next, we present a robust falling strategy for real humanoids that can manage
a wide range of perturbations by planning the optimal contact sequences.
We then introduce an iterative learning framework to intuitively design various
agile motions, which is inspired by human learning techniques.
The proposed iterative framework is followed by novel algorithms to
efficiently optimize control parameters for the target tasks, especially when
they have many constraints or parameterized goals.
Finally, we introduce an iterative approach for exporting simulation-optimized
control policies to hardware of robots to reduce the number of hardware
experiments, that usually accompany expensive costs and labors.





