\section{Motivation}
%% \karen{First paragraph: Motivating the importance of fall recovery for
%%   biped humanoids by describing how falls usually happen and what
%%   types of damages they cause. What are the hardware solutions in
%%   practice? Harness? Structural support? What are the drawbacks?}
A humanoid in an interactive environment is often exposed to the
risk of falling due to unexpected contacts or perturbations. A fall
can potentially cause detrimental damage to the robot and enormous
cost to repair. To reduce the likelihood of damaging robots during
online operations, researchers and engineers often cover the robot
exterior with soft guards to absorb the impact of falls
\cite{Missura:2011:DEH,Kobayashi:2011:DHD}. Though practical, these
extra parts can potentially limit the range of motion or change the
contact behaviors.

%% \karen{Second paragraph: An alternative approach inspiring by humans
%%   is for robots to learn basic motor skills to recover from
%%   falls. What are the existing papers in this area? Can be more
%%   thorough since we are not going to have a separate related work
%%   section.}
Alternatively, the robot can learn how to fall safely like humans
do. Fujiware \etal
\cite{Fujiwara:2002:UFM,Fujiwara:2003:FHH,Fujiwara:2004:SKL,Fujiwara:2006:TOF,Fujiwara:2007:OPF}
proposed fall strategies inspired by \emph{Ukemi}, a set of techniques
used in \emph{Judo}.  Ogata \etal \cite{Ogata:2007:FMC,Ogata:2008:RSG}
evaluated the risk of falls using predicted the zero moment point
(ZMP) and optimized the center of mass trajectory to reduce damage.
Ruiz-del-Solar \etal \cite{Ruiz:2009:LTF,Ruiz:2010:FDM} designed
falling sequences for simulated robots playing soccer.
%% Wang \etal \cite{Wang:2012:WTO} formulated an optimization of whole body
%% trajectories as a nonlinear programming problem and solved it with
%% heuristics \karen{this one doesn't say anything.}. 
Wang \etal \cite{Wang:2012:WTO} directly solved joint trajectories
of a three-link robot subject to the full-body dynamics.
Lee and Goswami \cite{Lee:2012:FOB} proposed a control strategy that
reorients the robot to fall on its backpack.
Yun and Goswami \cite{Yun:2014:TFC} described a ``Tripod'' strategy that
utilizes a swing foot and two hands to stop the fall with an
elevated center of mass.  Goswami \etal \cite{Goswami:2014:DCF}
proposed a direction-changing fall control strategy that utilizes
foot placement and inertia shaping to protect both the 
robot and objects/humans in the surroundings.
  
%% \karen{Third paragraph: These existing methods are designed for specific
%%   scenarios. Switching between methods requires an additional decision
%%   maker and carefully designed conditions. We hypothesize that there
%%   exists a continuous space of fall recovery strategies that can be
%%   characterized by the sequence of contact positions and the total
%%   number of contacts. As such, we view existing methods as special cases
%%   in this space. For example, stepping, tripod, Judo, etc...}
These existing methods are effective for specific scenarios in which
the perturbations are assumed within certain range. To select the best
method for the given scenario, an additional decision making process
that classifies initial conditions is required. In this chapter, we
hypothesize that there exists a continuous space of falling strategies
which can be characterized by the sequence of contact positions on the
robot. In this hypothesized space, the existing falling techniques can
be viewed as special cases. For example, the strategy proposed by
Ogata \etal~\cite{Ogata:2007:FMC} employed a single contact at
hands. Fujiwara \etal \cite{Fujiwara:2007:OPF} proposed to use two
contacts, knees and hands, to stop a fall with higher initial
momentum. In \emph{Judo}, multiple contact points from shoulders to
hips are used during a forward roll to break a high-speed fall \cite{ZenpoUkemi:2014:URL}. 

% \updated{These existing methods are designed for specific scenarios including
%   directions and magnitudes of perturbations.
%   Therefore, an additional decision layer that classifies initial conditions
%   is required to select the best strategy for the given scenario.
%   We hypothesize that there exists a continuous space of fall recovery
%   strategies that can be characterized by the sequence of contact positions and
%   the total number of contacts.
%   In this space, we view the existing falling techniques as special cases.
%   For example, the strategy proposed by Ogata \etal~\cite{Ogata:2007:FMC} uses
%   a single contact at hands. 
%   The larger initial momentum requires more contacts, such as knees and hands
%   \cite{Fujiwara:2007:OPF} or a foot and hands \cite{Yun:2014:TFC}.
%   In \emph{Judo}, a forward roll from shoulders to hips is used when being
%   thrown forward \cite{ZenpoUkemi:2014:URL}. } 

  
%% \karen{Fourth paragraph: Based on this hypothesis, we introduce a
%%   general fall recovery control algorithm which unifies all the individual
%%   strategies previously proposed. Details on the key ideas.}
We introduce a general algorithm that unifies the existing techniques
for falling strategies. Our algorithm reacts to a wide range of
initial perturbations by automatically determining the total number of
contacts, the order of contacts, and the position and timing of
contacts. The sequence of contact is optimized such that the initial
momentum is dissipated with minimal damage on the robot. We introduce
an abstract model, which consists of an inverted pendulum and a
telescopic ``stopper'', to approximate the reactive motion when humans
fall. The efficient optimization is achieved by recursive planning in
the space of abstract model using dynamic programming. We demonstrate
that our algorithm plans various falling strategies with different
contact sequences on simulated humanoids and on the actual hardware.

% \updated{Based on the hypothesis, we introduce a general damage reduction
%   algorithm for humanoid falls that provides a unified framework for existing
%   falling strategies. 
%   For the given initial state and possible contacts, our algorithm produces
%   the optimal sequence of contacts and target poses that can break a
%   fall with minimal impulses. 
  %% To make a search in the infinite contact space feasible, our algorithm finds
  %% a sequence in a user-provided contact candidate graph that defines potential
  %% transtions  between contacts, such as ``feet to knees'' or ``hips to hands''.  
  % The efficient optimization is achieved by recursive planning in the space of
  % abstract model using dynamic programming.
  % We demonstrate that our algorithm produces various falling strategies with
  % different contact sequences for a wide range of perturbations.}

