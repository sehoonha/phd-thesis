\section{The Problem}
Consider a biped humanoid on the ground with an unstable initial state
due to a large initial velocity. If the robot cannot recover its
balance, what is the least damaging way to fall on the ground? It is
well known that the damage incurred at the instance of impact is
mainly due to the sudden change of momentum, which requires a large
impulse applied in a very short period of time. To completely stop a
fall, the robot has no choice but to absorb the initial momentum in
its entirety. However, the change of momentum needs not to happen so
suddenly. With an ideal control policy, the robot should be able to
reduce the magnitude of the impulse at peak by distributing one large
impulse to multiple smaller impulses over multiple contacts with the
ground.

In the discretized time domain, we define the
\emph{instantaneous impulse} at each time step $n$ as 
\begin{equation}
j_n = \int_{hn}^{h(n+1)} f_y(t) dt = h f_y(hn)
\end{equation} 
where $h$ is the
discretized time interval and $f_y(t)$ is the sum of the vertical
contact forces at time $t$. Because the robots considered in this
work are made of hard materials, the contacts between the robot and
the ground can be approximated as collisions between two ideal rigid
bodies. With this assumption, the largest instantaneous impulse during
each contact period typically occurs at the \emph{impact moment}, the
instance when the contact first establishes. The maximum impulse for
the entire falling process can then be defined as
$\max j_n, \; n\in\mathcal{T}$, where $\mathcal{T} = \{\hat{t}^{(i)}
| i = 1 \cdots k\}$. We denote an impact moment for the contact $i$
as $\hat{t}^{(i)}$ and the total number of contacts as $k$. Using this
expression, the goal of our problem is to find a sequence of contact
locations on the robot and their timing, such that $\max j_n$ is
minimized.
