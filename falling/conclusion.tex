\section{Conclusion}

We presented a general algorithm to minimize the damage of humanoid
falls by utilizing multiple contact points. For an initial state with
arbitrary planar momentum, our algorithm optimizes the contact
sequence using abstract models and dynamic programming. Unlike
previous methods, our algorithm automatically determines the total
number of contacts, the order of contacts, and the position and timing
of contacts, such that the initial momentum is dissipated with minimal
damage to the robot.

\revised{The discovered optimal falling strategies in this paper may not be
  identical to the strategies of real humans \cite{Judoukemi:2015:URL} due to
  the different joint structures or mass distributions.
  For instance, BioloidGP and a human take different contact sequences during a
  forward roll. 
  A roll of a human changes contacts continuously from shoulders to hips, while a
  roll of BioloidGP makes discrete contacts at its head and left foot
  (\figref{falling_motions}).
  It can be due to the simplicity of the input contact graph, or the existence of
  a flexible spine that helps continuous transitions of contacts. 
  For another case when the fall is initiated from the two feet stance, 
  our strategy finds knees and hands as optimal but a falling strategy of Judo
  proposes to break a forward fall with only both hands. 
  The reason can be that humans try to avoid to make contacts at joints,
  which are more fragile than limbs. }

% For future consideration, we plan on implementing a realtime falling
% controller on a small humanoid, BioloidGP. First, the realtime
% control needs to be achieved by caching the cost-to-go function for a
% wide range of states. The robot also requires sensors to identify the
% state in realtime, such as additional IMUs or contact sensors.
