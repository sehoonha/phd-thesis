Teaching a virtual character a new motor skill requires designing a
new controller. 
Designing a motor controller for a robot or a physically simulated
character requires much expertise and trial-and-errors. 
In contrast, teaching a real person a new motor skill
is a gradual and interactive process via instructions, demonstration,
and feedback.
Learning is a gradual process with interleaving coaching and practicing steps. The coach gives high-level instructions, and the trainee interprets the instructions via many practice trials. Based on the trainee's updated skill, the coach gives new instructions to incrementally refine the motion. This cycle repeats until the trainee's skill reaches the coach's requirements.

We believe that designing a controller can be largely improved by
methodologies used by real humans. 
When humanlike learning can be done efficiently, controller design
becomes much more easily for everyone and the controller can be easily
adapted, extended, and concatenated; very much like how humans
internalize learned motor skills for new tasks.

Our goal is to come up with an algorithmic framework to resemble human
learning process, such that the character can easily acquire new
skills or concatenate existing skills. Our approach is to design
algorithms capturing the iterative process of coaching and
practicing. We focus on two aspect of human learning behaviors. First,
during coaching stage, the character learns motor skills from
high-level instructions, such as ``extend legs'' or ``lean forward'',
rather than specific description of joint angles. Second, during
practicing stage, the character accumulates experience via trial and
error. Our algorithm particularly utilizes human's ability to learn
from failures. For example, falling on the ground or hitting obstacles
when learning to jump or vault can be very valuable experience in the
learning process.




Designing a motor controller for a robot or a physically simulated
character requires much expertise and trial-and-errors. In most
practical cases, the controllers developed for a particular task, say
walking, do not help other similar tasks, say running. In contrast,
humans learn motor skills via a gradual and accumulative process. Once
a simpler skill is acquired, more complex skills can be built upon
it. 
We believe that designing a controller can be largely improved by
methodologies used by real humans. 

When humanlike learning can be done efficiently, controller design
becomes much more easily for everyone and the controller can be easily
adapted, extended, and concatenated; very much like how humans
internalize learned motor skills for new tasks.

Our approach allows the user to train a virtual character via
high-level, human-interpretable instructions. Through iterative
process of coaching and practicing, the character incrementally refines
its motor skill. Because our algorithm captures important aspects
of human learning process, the character can learn more efficiently
without depending on motion data, parameter tuning, or massive amount
of simulation.



goal:
Design motor controllers from human instructions

Model the process of human learning motor skills

approach: 
Design algorithms analogous to iterative learning process via coaching
and practicing steps

How do toddlers learn to walk?
How do athletes learn to improve their skills?
Learning motor skills is an accumulative and gradual process. Two
examples. Developing motor controllers for robots or physically
simulated characters follows 

Motor controllers designed for robots or characters are carefully crafted by programmers via parameter tuning and trial simulation. They simulate the end results, but they do not capture the process of learning itself.
Consequently, these learned controllers operate independently, rather than becoming an internal repertoire of motor skills that can be adapted, refined, or concatenated. 

The methodologies used to design a controller are very different from the way humans learn motor skills.
Learning is a gradual process with interleaving coaching and practicing steps. The coach gives high-level instructions, and the trainee interprets the instructions via many practice trials. Based on the trainee's updated skill, the coach gives new instructions to incrementally refine the motion. This cycle repeats until the trainee's skill reaches the coach's requirements.

Our goal is to come up with an algorithmic framework to resemble human
learning process, such that the character can easily acquire new
skills or concatenate existing skills. Our approach is to design
algorithms capturing the iterative process of coaching and
practicing. We focus on two aspect of human learning behaviors. First,
during coaching stage, the character learns motor skills from
high-level instructions, such as ``extend legs'' or ``lean forward'',
rather than specific description of joint angles. Second, during
practicing stage, the character accumulates experience via trial and
error. Our algorithm particularly utilizes human's ability to learn
from failures. For example, falling on the ground or hitting obstacles
when learning to jump or vault can be very valuable experience in the
learning process.


Technical details


Learning from failures: In reality, failing has a high cost of injury and pain. Each failure experience is thus very valuable. The trainee quickly build a mental model to categorize the bad motor commands and the resulting failures. 


Monday:
toy example section
instruction interpreter
citations

Tuesday:
edit section 1-6
proofread assignment
write conclusion section

Wednesday:
edit result section

Thursday:
incorporate feedback

overview figure
result figures
teaser figure
CMA examples figures